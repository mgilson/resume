% !TEX TS-program = xelatex
% !TEX encoding = UTF-8 Unicode
% -*- coding: UTF-8; -*-
% vim: set fenc=utf-8

%%%%%%%%%%%%%%%%%%%%%%%%%%%%%%%%%%%%%%%%%%%%%%%%%%%%%%%%%%%%%%%%%
%% CV.tex
%% <https://github.com/zachscrivena/simple-resume-cv>
%% This is free and unencumbered software released into the
%% public domain; see <http://unlicense.org> for details.
%%%%%%%%%%%%%%%%%%%%%%%%%%%%%%%%%%%%%%%%%%%%%%%%%%%%%%%%%%%%%%%%%

% See "README.md" for instructions on compiling this document.

\documentclass[letterpaper,MMMyyyy,nonstopmode]{simpleresumecv}
% Class options:
% a4paper, letterpaper, nonstopmode, draftmode
% MMMyyyy, ddMMMyyyy, MMMMyyyy, ddMMMMyyyy, yyyyMMdd, yyyyMM, yyyy

%%%%%%%%%%%%%%%%%%%%%%%%%%%%%%%%%%%%%%%%%%%%%%%%%%%%%%%%%%%%%%%%%
%% PREAMBLE.
%%%%%%%%%%%%%%%%%%%%%%%%%%%%%%%%%%%%%%%%%%%%%%%%%%%%%%%%%%%%%%%%%

% CV Info (to be customized).
\newcommand{\CVAuthor}{Matthew Gilson, Ph.D}
\newcommand{\CVTitle}{Matt's CV.}
\newcommand{\CVNote}{CV compiled on {\today}}
\newcommand{\CVWebpage}{https://github.com/mgilson/resume}

% PDF settings and properties.
\hypersetup{
pdftitle={\CVTitle},
pdfauthor={\CVAuthor},
pdfsubject={\CVWebpage},
pdfcreator={XeLaTeX},
pdfproducer={},
pdfkeywords={},
unicode=true,
bookmarks=true,
bookmarksopen=true,
pdfstartview=FitH,
pdfpagelayout=OneColumn,
pdfpagemode=UseOutlines,
hidelinks,
breaklinks}

% Shorthand.
\newcommand{\Code}[1]{\mbox{\textbf{#1}}}
\newcommand{\CodeCommand}[1]{\mbox{\textbf{\textbackslash{#1}}}}

%%%%%%%%%%%%%%%%%%%%%%%%%%%%%%%%%%%%%%%%%%%%%%%%%%%%%%%%%%%%%%%%%
%% ACTUAL DOCUMENT.
%%%%%%%%%%%%%%%%%%%%%%%%%%%%%%%%%%%%%%%%%%%%%%%%%%%%%%%%%%%%%%%%%

\begin{document}

%%%%%%%%%%%%%%%
% TITLE BLOCK %
%%%%%%%%%%%%%%%

\Title{\CVAuthor}

\begin{SubTitle}
\href{https://www.google.com/maps/place/1400+Hopkins+Avenue,+Redwood+City,+California+994062,+USA}
{1400 Hopkins Ave, Apt 303, Redwood City, CA 94062, USA}
\par
\href{mailto:m.gilson1@gmail.com}
{m.gilson1@gmail.com}
\,\SubBulletSymbol\,
(603)\,892-7736
\end{SubTitle}

\begin{Body}

%%%%%%%%%%%%%%%
%% EDUCATION %%
%%%%%%%%%%%%%%%

\Section
{Education}
{Education}
{PDF:Education}

\Entry
\href{http://www.unh.edu/}
{\textbf{University of New Hampshire}},
Durham, New Hampshire, USA

\Gap
\BulletItem
Ph.D. in
\href{http://physics.unh.edu/content/physics-phd}
{Physics and Space Science}
\hfill
\DatestampYM{2007}{08} --
\DatestampYM{2011}{12}
\begin{Detail}
\SubBulletItem
Thesis: \textit{Global Structure of the Nightside Proton Precipitation During Substorms Using Simulations and Observations}
\SubBulletItem
Adviser:
Prof.~Jimmy~Raeder
\end{Detail}

\BigGap
\Entry
\href{http://www.gcc.edu/Pages/Grove-City-College.aspx}
{\textbf{Grove City College}},
Grove City, Pennsylvania, USA

\Gap
\BulletItem
B.S. in
\href{http://www.gcc.edu/academics/SEM/physics/Pages/default.aspx}
{Applied Physics}
\hfill
\DatestampYM{2003}{08} -- \DatestampYM{2007}{05}

\BigGap
\Entry
\href{https://www.udacity.com}
{\textbf{Udacity}}

\Gap
\BulletItem
Nanodegree in
\href{https://www.udacity.com/course/machine-learning-engineer-nanodegree--nd009}
{Machine learning} (in progress)
\hfill
\DatestampYM{2016}{09} -- Present


%%%%%%%%%%%%%%%%%%%%%%%%%
%% Work EXPERIENCE %%
%%%%%%%%%%%%%%%%%%%%%%%%%

\Section
{Work Experience}
{Work Experience}
{PDF:ResearchExperience}

\Entry
\href {https://getpattern.com}{\textbf{Pattern Technologies Inc.}}
\Gap
\BulletItem
Software Engineer
\hfill
\DatestampYM{2015}{07} -- Present
\begin{Detail}
\SubBulletItem
Joined the founding team at an early stage startup.  Developed automated tooling
for code testing and deployment.  Coded the product's web application and
marketing websites along side two other engineers that served hundreds of customers daily.
\end{Detail}

\BigGap
\Entry
\href {https://google.com}{\textbf{Google Inc.}}
\Gap
\BulletItem
Web Solutions Engineer
\hfill
\DatestampYM{2015}{07} -- \DatestampYM{2013}{07}
\begin{Detail}
\SubBulletItem
Developed internal tools to support Google's business.
\SubBulletItem
Conducted code reviews for my team and others.
\end{Detail}

\BigGap
\Entry
\href {http://www.unh.edu/}{\textbf{University of New Hampshire}}
\Gap
\BulletItem
Research Scientist
\hfill
\DatestampYM{2012}{01} -- \DatestampYM{2013}{07}
\begin{Detail}
\SubBulletItem
Coupled the global MHD Magnetospheric simulation (OpenGGCM) with the Rice Convection
Model of the inner magnetosphere.  This enabled more accurate simulation of the
Earth's magnetic field and it's interactions with solar plasmas.
\SubBulletItem
Published peer reviewed journal articles.
\SubBulletItem
Mentored graduate students.
\end{Detail}


%%%%%%%%%%%%%%%%%%%%%%%%%%%%%%
%% AWARDS & ACCOMPLISHMENTS %%
%%%%%%%%%%%%%%%%%%%%%%%%%%%%%%

\Section
{Awards \&\newline
Achievements}
{Awards \& Achievements}
{PDF:AwardsAndAchievements}

\Entry
\textbf{StackOverflow}
\hfill
\DatestampY{2011} -- Present
\begin{Detail}
\Item
Answered nearly 4000 questions.
\Item
Top \href{http://stackoverflow.com/users/748858/mgilson}{0.06\% all time.}
\Item
10th all time leading answerer of \href{http://stackoverflow.com/tags/python/topusers}{python questions.}
\end{Detail}


%%%%%%%%%%%%
%% SKILLS %%
%%%%%%%%%%%%

\Section
{Skills}
{Skills}
{PDF:Skills}

\Entry
\textbf{Programming Languages}
\begin{Detail}
\Item Python (fluent), Javascript, C (proficient), Fortran (fluent), Bash (proficient), Java (novice)
\end{Detail}

\BigGap
\Entry
\textbf{Frameworks}
\begin{Detail}
\Item Google App Engine, AngularJS
\end{Detail}

\BigGap
\Entry
\textbf{Agile Developement}
\begin{Detail}
\Item Certified Scrum Master in \DatestampY{2015}.  Still practice agile methodologies
at current workplace.
\end{Detail}

% %%%%%%%%%%%%%%%
% %% INTERESTS %%
% %%%%%%%%%%%%%%%

% \Section
% {Interests}
% {Interests}
% {PDF:Interests}

% \Entry
% Programming, Science (particularly Physics), High Performance Computing, Machine Learning, Rock Climbing

%%%%%%%%%%%%%%%%%%%%%%%%%%%
%% SELECTED PUBLICATIONS %%
%%%%%%%%%%%%%%%%%%%%%%%%%%%

\Section
{Selected Publications}
{Selected Publications}
{PDF:Selected Publications}

\SubSection
{Journals}
{Journals}
{PDF:Journals}

\BigGap
\href{http://onlinelibrary.wiley.com/doi/10.1029/2011JA016640/abstract}
{\underline{Gilson~M.~L.}, J.~Raeder, E.~F.~Donovan, Y.~S.~Ge and S.~B.~Mende (2011)
``Statistics of the longitudinal splitting of proton aurora during substorms'',
\textit{J. Geophys. Res., 116}(A8), A08,206, doi:10.1029/2011JA016640}

\BigGap
\href{http://onlinelibrary.wiley.com/doi/10.1029/2012JA017562/full}
{\underline{Gilson,~M.~L.}, J.~Raeder, E.~Donovan, Y.~S.~Ge, and L.~Kepko (2012), ``Global simulation of proton precipitation due to field line curvature during substorms",
\textit{J. Geophys. Res., 117}, A05216, doi:10.1029/2012JA017562.}

\BigGap
\href{http://onlinelibrary.wiley.com/doi/10.1002/2017JA024104/full}
{Cramer~W.D., J.~Raeder, F.R.~Toffoletto, \underline{M.L~Gilson}, and B.~Hu (2017), ``Plasma Sheet Injections into the Inner Magnetosphere: Two-Way Coupled OpenGGCM-RCM Model Results'',
\textit{J. Geophys. Res., 122}, doi:10.1002/2017JA024104.}



% % Manual page break.
% \newpage

% %%%%%%%%%%%%%%%%%%%%%%%%%%%%%%%%%%%%%%%%
% %% THIS IS A SECTION WITH USAGE NOTES %%
% %%%%%%%%%%%%%%%%%%%%%%%%%%%%%%%%%%%%%%%%

% % Declare a new group to limit the scope of \color to this section.
% \begingroup
% \color{red}

% \Section
% {This is a\newline
% Section\newline
% With\newline
% Usage Notes}
% {This is a Section With Usage Notes (For PDF Bookmark)}
% {PDF:ThisIsASectionWithUsageNotes:ForPDFLink}

% \SubSection
% {This is a SubSection}
% {This is a SubSection (For PDF Bookmark)}
% {PDF:ThisIsASubSection:ForPDFLink}

% \BigGap
% \BulletItem
% Use \CodeCommand{Section\{a\}\{b\}\{c\}} and
% \CodeCommand{SubSection\{a\}\{b\}\{c\}}
% to create sections and subsections, where
% \Code{a} is the heading displayed on the page,
% \Code{b} is the PDF bookmark heading, and
% \Code{c} is the internal PDF link (must be unique).
% Sections and subsections will appear in the PDF bookmarks.
% Note the CamelCase command names.

% \Gap
% \BulletItem
% Use
% \CodeCommand{Entry},
% \CodeCommand{BulletItem},
% \CodeCommand{SubBulletItem},
% \CodeCommand{Item},
% \CodeCommand{SubItem},
% \CodeCommand{NumberedItem},
% etc.,
% to create entries in the main body of the CV.

% \Gap
% \BulletItem
% Enclose entry details between
% \CodeCommand{begin\{Detail\}} and
% \CodeCommand{end\{Detail\}}
% so that they are typeset in a smaller font.
% \begin{Detail}
% \Item
% This is an example of entry detail text enclosed in a \Code{Detail} environment.
% \end{Detail}

% \Gap
% \BulletItem
% Use \CodeCommand{Gap} and \CodeCommand{BigGap} to insert vertical spaces between entries to improve layout.

% \BigGap
% \SubSection
% {This is Another SubSection}
% {This is Another Subsection (For PDF Bookmark)}
% {PDF:ThisIsAnotherSubSection:ForPDFLink}

% \BigGap
% \Entry
% This is a plain \CodeCommand{Entry},
% followed by an \CodeCommand{hfill} and a date range
% \hfill
% \DatestampYM{2015}{10} --
% \DatestampYM{2015}{12}

% \Gap
% \BulletItem
% This is a \CodeCommand{BulletItem}.
% \Item
% This is an \CodeCommand{Item}, which has no bullet.
% Note the alignment with the \CodeCommand{BulletItem} above.

% \Gap
% \SubBulletItem
% This is a \CodeCommand{SubBulletItem}.
% \SubItem
% This is a \CodeCommand{SubItem}, which has no bullet.
% Note the alignment with the \CodeCommand{SubBulletItem} above.

% \Gap
% \NumberedItem{[42]}
% This is a \CodeCommand{NumberedItem}.
% Change the value of the macro \CodeCommand{MaxNumberedItem} to adjust the indentation width.

% \BigGap
% \SubSection
% {Line, Paragraph, and Page Breaks}
% {Line, Paragraph, and Page Breaks (For PDF Bookmark)}
% {PDF:LineParagraphAndPageBreaks:ForPDFLink}

% \BigGap
% \BulletItem
% To create a new line within the same paragraph (i.e., preserving the same paragraph indentation), use \CodeCommand{newline} instead of \CodeCommand{\textbackslash};
% the latter will reset the paragraph indentation.

% \Gap
% \BulletItem
% To create a new paragraph, use \CodeCommand{par} or simply leave an empty line.
% Paragraph indentations (from
% \CodeCommand{Entry},
% \CodeCommand{BulletItem},
% \CodeCommand{SubBulletItem},
% \CodeCommand{Item},
% \CodeCommand{SubItem},
% \CodeCommand{NumberedItem},
% etc.) do not carry across different paragraphs.

% \Gap
% \BulletItem
% To create a new page, use \CodeCommand{newpage}.

% \BigGap
% \SubSection
% {Dates}
% {Dates (For PDF Bookmark)}
% {PDF:Dates:ForPDFLink}

% \BigGap
% \BulletItem
% Use the following macros to specify and display dates consistently:
% \SubBulletItem
% \CodeCommand{DatestampYMD\{yyyy\}\{MM\}\{dd\}}
% (e.g., \CodeCommand{DatestampYMD\{2008\}\{01\}\{15\}})
% \SubBulletItem
% \CodeCommand{DatestampYM\{yyyy\}\{MM\}}
% (e.g., \CodeCommand{DatestampYM\{2008\}\{01\}})
% \SubBulletItem
% \CodeCommand{DatestampY\{yyyy\}}
% (e.g., \CodeCommand{DatestampY\{2008\}})

% \Gap
% \BulletItem
% Change the date format option passed to the document class to adjust how dates are displayed throughout the document:
% \SubBulletItem
% \Code{MMMyyyy} (``Jan~2008'')
% \SubBulletItem
% \Code{ddMMMyyyy} (``15~Jan~2008'')
% \SubBulletItem
% \Code{MMMMyyyy} (``January~2008'')
% \SubBulletItem
% \Code{ddMMMMyyyy} (``15~January~2008'')
% \SubBulletItem
% \Code{yyyyMMdd} (``2008-01-15'')
% \SubBulletItem
% \Code{yyyyMM} (``2008-01'')
% \SubBulletItem
% \Code{yyyy} (``2008'')

% \endgroup

\end{Body}

%%%%%%%%%%%
% CV NOTE %
%%%%%%%%%%%

% \BigGap
% \UseNoteFont%
% \null\hfill%
% [\textit{\CVNote}]

\end{document}
